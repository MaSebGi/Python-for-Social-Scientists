
\documentclass{article}

\usepackage{graphicx}
%\usepackage[altbullet]{lucidabr}
%two lines below change font (font intalled manually (i.e. uploaded))
%\usepackage{fontspec}
%\setmainfont[Ligatures=TeX]{LucidaBrightRegular.ttf}
%\usepackage{kpfonts}    % for nice fonts
% option [light] for more aery documents
\usepackage{color}  %for color of references
\usepackage[dvipsnames]{xcolor} %for color of references
\usepackage{caption}
\usepackage{fancyhdr}
\usepackage[pagebackref,colorlinks, citecolor=Brown,urlcolor=Brown]{hyperref}
\usepackage{natbib}
\usepackage{multicol}
\usepackage{multirow}
%\usepackage{lscape}
\usepackage{pdflscape}
\usepackage{amssymb}
\usepackage{geometry}
\usepackage{longtable}
\usepackage{colortbl}
\usepackage{dsfont}
\usepackage{bm}
\usepackage{mathtools}
\usepackage{pgf}
\usepackage{tikz}
\usepackage{soul}
\usepackage{tikz}
\usepackage{tikz,fullpage}
\usepackage{pgf}
\usepackage{tikz}
\usepackage{bold-extra} %for bold small caps in the title

\renewcommand{\familydefault}{\sfdefault} %for the sans serif font

\newtheorem{theorem}{Theorem}[section]
\newtheorem{lemma}[theorem]{Lemma}
%\theoremstyle{definition}
\newtheorem{definition}[theorem]{Definition}
\newtheorem{example}[theorem]{Example}
\newtheorem{xca}[theorem]{Exercise}
%\theoremstyle{remark}
\newtheorem{remark}[theorem]{Remark}
\numberwithin{equation}{section}

%    Absolute value notation
\newcommand{\abs}[1]{\lvert#1\rvert}

%    Blank box placeholder for figures (to avoid requiring any
%    particular graphics capabilities for printing this document).
\newcommand{\blankbox}[2]{%
  \parbox{\columnwidth}{\centering
%    Set fboxsep to 0 so that the actual size of the box will match the
%    given measurements more closely.
    \setlength{\fboxsep}{0pt}%
    \fbox{\raisebox{0pt}[#2]{\hspace{#1}}}%
  }%
}

\begin{document}

\title{\Large Python for Social Scientists}

%    Information for first author
%\author{Author One}
%\address{}
%\curraddr{}
%\email{}
%\thanks{}

%    Information for second author
%\author{}
%\address{}
%\email{}
%\thanks{}

%    General info
%\subjclass[2000]{}

\date{\vspace{-1cm}}

%\date{Current version \today. Initial draft: July, 2019. \\
%Prepared by Arnaud Dy\`evre (\href{
%mailto:a.dyevre@lse.ac.uk}{a.dyevre@lse.ac.uk}) and Jialin Yi (\href{
%mailto:j.yi8@lse.ac.uk}{j.yi8@lse.ac.uk})}

%\dedicatory{}
%\keywords{}

%\begin{abstract}

%\end{abstract}

\maketitle

\begin{center}
    September 16\textsuperscript{th}-27\textsuperscript{th}, 2019 \\
    PhD Academy - LSE
\end{center}

\vspace{1cm}

%\section*{This is an unnumbered first-level section head}
%This is an example of an unnumbered first-level heading.

%% The correct journal style for \specialsection is all uppercase; a known bug
%% in amsart.cls prevents this, so input must be uppercase until it is fixed.
%\specialsection*{This is a Special Section Head}
%\specialsection*{THIS IS A SPECIAL SECTION HEAD}
%This is an example of a special section head%
%%%%%%%%%%%%%%%%%%%%%%%%%%%%%%%%%%%%%%%%%%%%%%%%%%%%%%%%%%%%%%%%%%%%%%%%
%\footnote{Here is an example of a footnote. Notice that this footnote text is running on so that it can stand as an example of how a footnote with separate paragraphs should be written.
%\par
%And here is the beginning of the second paragraph.}%
%%%%%%%%%%%%%%%%%%%%%%%%%%%%%%%%%%%%%%%%%%%%%%%%%%%%%%%%%%%%%%%%%%%%%%%%


%\section{Applications: Consumption Responses to Income \& Wealth Shocks}

\subsection*{Practical details}

\vspace{0.5cm}

\begin{tabular}{lll}
    \textbf{Dates} &  September 16\textsuperscript{th}-27\textsuperscript{th} & (no class on the 25\textsuperscript{th})\\
    \textbf{Time} &  10:00am-1:00pm & \\
    \textbf{Place} & PhD Academy training room & \\
        & Lionel Robbins building, 4\textsuperscript{th} floor & \\
    \textbf{Instructor} & Jialin Yi & \\
     & PhD candidate, Department of Statistics, LSE & \\
     & \href{
mailto:j.yi8@lse.ac.uk}{j.yi8@lse.ac.uk} & \\
\end{tabular}

\vspace{0.5cm}

\subsection*{Course Description}

\vspace{0.5cm}

This two-week course will provide an overview of the tools and methods required to undertake a collaborative research project in Python. It is designed for first- and second-year PhD students in the social sciences with the ambition to conduct quantitative research in Python. No knowledge of Python is required to take this course, although familiarity with elements of programming will help students get the most out of it. The class material was created for an audience familiar with Stata, MATLAB or R, looking to transition away from proprietary softwares and to be able to undertake all aspects of a research project within a single programming environment. \\

By the end of the two weeks, students will be able to use GitHub to manage a collaborative research project, to use most of the econometrician's standard tools in Python, and they will have gained familiarity with standard machine learning techniques.

\vspace{0.5cm}

\subsection*{Prerequisites}

\vspace{0.5cm}

Previous knowledge of Python is not required but will greatly help. We expect the course to be fast-paced and the instructor's time will be better used to help students assimilating the class material. If you are unfamiliar with Python, we recommend to go through the three first lectures of \href{https://lectures.quantecon.org/py/}{QuantEcon}: \href{https://lectures.quantecon.org/py/index_learning_python.html}{``Introduction to Python''}, \href{https://lectures.quantecon.org/py/index_python_scientific_libraries.html}{``the Scientific Libraries''}, and \href{https://lectures.quantecon.org/py/index_advanced_python_programming.html}{``Advanced Python Programming''}.\footnote{QuantEcon is a website created by economists Thomas J. Sargent and John Stachurski, teaching social scientists how to use Python for research. Extensive accompanying lecture notes are available in .pdf format \citep{sargent2019lectures}.} No knowledge of version control, GitHub or machine learning is required.\\

While not a required reading, ``Code and Data for the Social Sciences'' by \cite{gentzkow2014code} is a very informative resource. This 40-page long paper describes the best practices for collaborative research projects in the social sciences. Chapters 1, 3, 6, 7 and the appendix on code style are worth reading before the class starts.\\

Students should be familiar with the various statistical tools whose implementation in Python will be demonstrated during the class (see the Course Outline section below).

\vspace{0.5cm}

\subsection*{Computation}

\vspace{0.5cm}

The course will consist of a combination of demonstrations and in-class exercises, so students are advised to bring their own laptops. If you need a laptop, the PhD Academy can provide you with an iRoam machine for the duration of the course.\\

Please install the latest version of Python (3.7) through the Anaconda distribution, ahead of the course. The iRoam laptops will already have Anaconda installed. Anaconda is a free and popular platform for data scientists working with Python. It comes with all the packages we will use in the class. You can download Anaconda from \href{https://www.anaconda.com/distribution/}{here}.

\vspace{0.5cm}

\subsection*{Course Outline}

\vspace{0.5cm}

[e] indicates class demonstrations or exercises\\

\hl{[This section will be subject to some changes as we design the course, please consider it as indicative only]} \\

\begin{center}
    \begin{tabular}{| p{2.5cm} | p{12.5cm} |}
    \hline
    \multicolumn{2}{|l|}{\textbf{Module 1: Version control for collaborative projects}} \\
    \multicolumn{2}{|l|}{3 days} \\
    \hline
       Monday 16  & \textit{Version control (1)} \\
      
            & \\
       Tuesday 17  & \textit{Version control (2)} \\
            & \\
       Wednesday 18 & \textit{Using third party online computing resources} \\
            & \\
    \hline
    \end{tabular}
\end{center}

\begin{center}
        \begin{tabular}{| p{2.5cm} | p{12.5cm} |}
    \hline
    \multicolumn{2}{|l|}{\textbf{Module 2: Statistics and Econometrics}} \\
    \multicolumn{2}{|l|}{5 days} \\
    \hline
    Thursday 19 & \textit{Basics of data handling} \\
        & \\
    Friday 20 & \textit{OLS, GLS, IV and NLLS}\\
        & \\
    Monday 23 & \textit{Maximum Likelihood and Limited Dependent Variable Models} \\
        & \\
    Tuesday 24 & \textit{Time Series}\\
        & \\
    Tuesday 25 & --No class--\\
        & \\
    Wednesday 26 & \textit{GMM} \\
        & \\
\hline
    \end{tabular}
\end{center}

\begin{center}
    \begin{tabular}{| p{2.5cm} | p{12.5cm} |}
    \hline
    \multicolumn{2}{|l|}{\textbf{Module 3: Elements of Machine Learning}} \\
    \multicolumn{2}{|l|}{1 day} \\
    \hline
    Thursday 27 & \textit{Introduction to the data scientist's toolkit} \\
        & \\
    \hline
    \end{tabular}

\end{center}

\begin{center}
    
\end{center}

\subsection*{Resources}

In class, the instructor will use Jupyter Notebooks for demonstrations, and students will use the Spyder scientific environment for hands-on exercises. Both can be used within Anaconda. All Notebooks, datasets, lecture notes and slides will be available on the course GitHub page \hl{[TBC]}.

\newpage 

\bibliographystyle{ecta}
\bibliography{PSSSyllabus}

\end{document}

%------------------------------------------------------------------------------
% End of journal.tex
%------------------------------------------------------------------------------
