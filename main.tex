%\documentclass{article}
\documentclass[12pt, titlepage]{article}
\usepackage[utf8]{inputenc}
%\usepackage[altbullet]{lucidabr}
%two lines below change font (font intalled manually (i.e. uploaded))
%\usepackage{fontspec}
%\setmainfont[Ligatures=TeX]{LucidaBrightRegular.ttf}
%\usepackage{kpfonts}    % for nice fonts
% option [light] for more slender documents
\usepackage{color}  %for color of references
\usepackage[dvipsnames]{xcolor} %for color of references
\usepackage{graphicx}
\usepackage{caption}
\usepackage{fancyhdr}
\usepackage[pagebackref,colorlinks, citecolor=Brown,urlcolor=Brown]{hyperref}
\usepackage{natbib}
\usepackage{multicol}
\usepackage{multirow}
%\usepackage{lscape}
\usepackage{pdflscape} %uncomment this  and comment above line to see the difference
\usepackage{amssymb}
\usepackage{geometry}
\usepackage{longtable}
\usepackage{colortbl}
\usepackage{dsfont}
\usepackage{bm}
\usepackage{mathtools}
\usepackage{pgf}
\usepackage{tikz}
\usepackage{soul}
\usepackage{tikz}
\usepackage{tikz,fullpage}
\usepackage{pgf}
\usepackage{tikz}
\usepackage{bold-extra} %for bold small caps in the title
\usetikzlibrary{arrows,automata}
\tikzstyle{selected edge} = [draw,line width=5pt,-,red!50]
\tikzstyle{ignored edge} = [draw,line width=5pt,-,black!20]

\tikzset{
    vertex/.style = {
        circle,
        fill            = black,
        outer sep = 2pt,
        inner sep = 1pt,
    }
}

\renewcommand{\headrulewidth}{0pt}
\textwidth15.5cm \textheight23cm \topmargin0.5cm \footskip1.5cm
\oddsidemargin-0.00cm
\newcommand{\hmax}{\overline{h}}
\renewcommand{\partname}{Leibniz}
\renewcommand\thepart{\arabic{part}}

\newenvironment{changemargin}[2]{%
\begin{list}{}{%
\setlength{\topsep}{0pt}%
\setlength{\leftmargin}{#1}%
\setlength{\rightmargin}{#2}%
\setlength{\listparindent}{\parindent}%
\setlength{\itemindent}{\parindent}%
\setlength{\parindent}{0cm}%
%setlength{\itemindent}{0cm}%
\setlength{\parsep}{\parskip}%
}%
\item[]}{\end{list}}
\setlength{\parindent}{0cm}%
%\renewcommand{\baselinestretch}{1.5}

\pagenumbering{arabic}

\title{Statement of Purpose}
\author{Arnaud Dyèvre}
\date{\today}


\pagestyle{fancy}
\fancyhf{}
\fancyhead[R]{}

\begin{document}

\begin{center}
\Large
\end{center}
%\vspace{+0.5cm}


\begin{center}
\textbf{\Large \textsc{Python for Social Scientists}} \\ \Large Syllabus\\

\end{center}

\begin{center}
\normalsize September 16\textsuperscript{th} to 26\textsuperscript{th}

\end{center}


\begin{center}

\end{center}

%\maketitle

\section*{Practical details}

\begin{center}
\begin{tabular}{ p{3cm}p{3cm}p{4.5cm}p{4.5cm} } 
\hline
 \textbf{Date} & \textbf{Time} & \textbf{Place} & \textbf{Intructor} \\
\hline
 September 16\textsuperscript{th}- & 10:00am-1:00pm & PhD Academy & Jialin Yi \\
 September 27\textsuperscript{th} & Everyday & training room, & PhD candidate,  \\
     &   & Lionel Robbins building &  Dept. of Statistics, LSE \\
    &   & 4\textsuperscript{th} floor &  \href{
mailto:j.yi8@lse.ac.uk}{j.yi8@lse.ac.uk} \\
\hline
\end{tabular}
\end{center}

\section*{Prerequisites}

Please read ``Code and Data for the Social Sciences''

\section*{Computation}

The course will consist of a combination of demonstrations and in-class exercises, so students are advised to bring their own laptops to get the most out of the course. If you need a laptop, the PhD Academy can provide you with an iRoam machine for the duration of the course.\\
Please install Python 3.7 \textit{Anaconda} distribution ahead of the course. The iRoam laptops will already have \textit{Anaconda} installed. \textit{Anaconda} is a popular platform for data scientists working with Python. It comes with all the packages we will use in the class. You can download \textit{Anaconda} from \href{https://www.anaconda.com/distribution/}{here}.

\section*{Course Outline}

\begin{center}
    \begin{tabular}{p{4cm} | p{10cm} }
    \hline
    \multicolumn{2}{l}{\textbf{Module 1:} \emph{Version control for collaborative projects}} \\
    \hline
       Monday 16  &  \\
       Tuesday 17  & \\
       Wednesday 18 & \\
    \hline
    \multicolumn{2}{l}{\textbf{Module 2:} \emph{Statistics and Econometrics}} \\
    \hline
    Thursday 19 & \\
    Friday 20 & \\
    Monday 23 & \\
    Tuesday 24 & \\
    Wednesday 25 & \\
    \hline
    \multicolumn{2}{l}{\textbf{Module 3:} \emph{Elements of Machine Learning}} \\
    \hline
    Thursday 26 & \\
    \hline
    \end{tabular}
\end{center}

\end{document}

